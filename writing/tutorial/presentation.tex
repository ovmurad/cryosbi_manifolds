\documentclass[aspectratio=169]{beamer}
% ======================================================================
% Theme & appearance
% ======================================================================
\usetheme{metropolis}

\metroset{
	progressbar=frametitle,
	numbering=fraction,
	titleformat=smallcaps,
}

\usepackage[T1]{fontenc}
\usepackage[utf8]{inputenc}
\usepackage{lmodern}

% ======================================================================
% Macros & Commands
% ======================================================================
\usepackage{amsmath, amsfonts, amssymb}

% ======================================================================
% Sets
% ======================================================================
\NewDocumentCommand{\R}{o}{\IfValueTF{#1}{\mathbb{R}^{#1}}{\mathbb{R}}}
% ======================================================================
% Comments
% ======================================================================
\NewDocumentCommand{\luke}{m}{{\color{blue}{LE}: #1}}
\NewDocumentCommand{\mmp}{m}{{\color{cyan}{MMP}: #1}}
\NewDocumentCommand{\vlad}{m}{{\color{violet}{OVM}: #1}}

% ======================================================================
% Algos
% ======================================================================
\NewDocumentCommand{\dm}{}{\textsc{DiffMaps} }
\NewDocumentCommand{\tslasso}{}{\textsc{TSLasso} }
\NewDocumentCommand{\ies}{}{\textsc{IES} }
\NewDocumentCommand{\rrelex}{}{\textsc{RRelax} }

% ======================================================================
% Operators
% ======================================================================
\NewDocumentCommand{\dk}{o O{k}}{\IfValueTF{#1}{\mathrm{dist}_{#2}\left(#1\right)}{\mathrm{dist}_{#2}}}
\NewDocumentCommand{\nr}{o O{r}}{\IfValueTF{#1}{\mathrm{n}_{#2}\left(#1\right)}{\mathrm{n}_{#2}}}

% ======================================================================
% Misc
% ======================================================================
\NewDocumentCommand{\codelink}{m}{\href{https://github.com/ovmurad/cryosbi_manifolds}{#1}}
\NewDocumentCommand{\hem}{}{hemagglutinin }
\NewDocumentCommand{\igg}{}{IgG }



% ======================================================================
% Metadata
% ======================================================================
\title{Manifold Learning and Simulation-Based Inference for Cryo-EM}
\author{Octavian-Vlad Murad, Luke Evans, Marina Meilă}
\institute{University of Washington, Flatiron Institute, University of Waterloo}
\date{\today}

% ======================================================================
% Presentation
% ======================================================================
\begin{document}


% ======================================================================
% Title, Contents, Acknowledgement, etc.
% ======================================================================
% ----------------------------------------------------------------------
\begin{frame}
	\titlepage
\end{frame}

% ----------------------------------------------------------------------
\begin{frame}{Outline}
	\tableofcontents
\end{frame}

% ======================================================================
\section{Motivation}
% ======================================================================
% ----------------------------------------------------------------------
\begin{frame}{Motivation}
	\begin{itemize}
		\item Examples of relevant SBI (Simulation-Based Inference) problems
		\item Our main Cryo-EM question
	\end{itemize}
\end{frame}

% ----------------------------------------------------------------------
\begin{frame}{Motivation for Manifold Learning (ML)}
	\begin{itemize}
		\item Many natural data sets are intrinsically low-dimensional
		\item Neural network embeddings are often low-dimensional as well
		\item Therefore: manifold learning can help in many ways (incomplete list):
		\begin{itemize}
			\item estimate dimension
			\item check whether simulated data covers real data
			\item reduce dimension via embedding algorithms
			\item visualize and interpret latent structure
			\item measure distortion
			\item perform SBI in a low-dimensional representation
		\end{itemize}
	\end{itemize}
\end{frame}

% ======================================================================
\section{Basics of ML with SuperMan}
% ======================================================================
% ----------------------------------------------------------------------
\begin{frame}{Basics of ML with SuperMan}
	\begin{itemize}
		\item Brief intro to manifold learning algorithms:
		\begin{itemize}
			\item Diffusion Maps
			\item t-SNE
		\end{itemize}
		\item Why ML is harder than PCA:
		\begin{itemize}
			\item many parameter choices (neighbors $k$, radius $\varepsilon$, metrics, etc.)
		\end{itemize}
		\item Hands-on SuperMan on a toy data set:
		\begin{itemize}
			\item Swiss roll with a hole
		\end{itemize}
	\end{itemize}
\end{frame}

% ----------------------------------------------------------------------
\begin{frame}{Implementation: SuperMan vs.\ scikit-learn}
	\begin{itemize}
		\item Performance:
		\begin{itemize}
			\item SuperMan: fast, real-time or semi-real-time
			\item scikit-learn: slower; may not complete on large data sets
		\end{itemize}
		\item Statistical and geometric features in SuperMan:
		\begin{itemize}
			\item neighbor selection ($k$-NN, radius $\varepsilon$)
			\item Riemannian metric estimation (RMetric)
			\item intrinsic dimension estimation (IES)
			\item other diagnostics and elementary statistics
		\end{itemize}
	\end{itemize}
\end{frame}

% ======================================================================
\section{SBI Validation Pipeline}
% ======================================================================
% ----------------------------------------------------------------------
\begin{frame}{SBI Validation Pipeline}
	\begin{itemize}
		\item Construct distance matrix and local statistics.
		\item Preprocessing I: Outlier removal.
		\item Assess coverage and fidelity.
		\item Preprocessing II: Uniform resampling.
		\item Construct the Laplacian and recompute local statistics.
		\item Estimate intrinsic dimension.
		\item Manifold Learning: \dm.
		\item Manifold Interpretation \& Visualization I: \ies and plotting.
		\item Manifold Interpretation \& Visualization II: \tslasso.
		\item Manifold Interpretation \& Visualization III: \rrelex.
	\end{itemize}
\end{frame}

% ----------------------------------------------------------------------
% 1. Construct distance matrix and local statistics
% ----------------------------------------------------------------------
\begin{frame}{Construct Distance Matrix and Local Statistics}
	
	\textbf{Optional Step: Sub-sampling for tractability}
	\begin{itemize}
		\item Many downstream operations scale superlinearly in $N$:
		\begin{itemize}
			\item pairwise distances / neighbor graphs,
			\item spectral decompositions of graph Laplacians,
			\item local statistics (kNN, radius counts, etc.),
			\item local covariance matrices and their spectral decompositions. 
		\end{itemize}
		\item If $N$ is very large:
		\begin{itemize}
			\item Randomly sub-sample to a size that:
			\begin{itemize}
				\item still captures the geometry of the data,
				\item fits within the available computational budget.
			\end{itemize}
			\item In general, one should use as many points as resources allow.
		\end{itemize}
	\end{itemize}
	
\end{frame}

\begin{frame}{Construct Distance Matrix and Local Statistics}
	
	\textbf{Main Step: Distance matrix and local statistics}
	\begin{itemize}
		\item Compute distances between latent points in $\X$.
		\item For each point $x \in \X$, compute local statistics:
		\begin{itemize}
			\item $\dk[x]$: distance to the $k$-th nearest neighbor for a range $k \in \{k_{\min}, \dots, k_{\max}\}$ or,
			\item $\nr[x]$: the number of neighbors within radius $r$ for a range $r \in \{r_{\min}, \dots, r_{\max}\}$.
		\end{itemize}
	\end{itemize}
	
	\textbf{Practical notes:}
	\begin{itemize}
		\item Store distances in sparse form.
		\item As we subsample the data, slice the already computed distance matrix.
		\item Reuse the same local statistics across multiple steps in the pipeline.
	\end{itemize}
\end{frame}


% ----------------------------------------------------------------------
% 2. Preprocessing I: Outlier removal
% ----------------------------------------------------------------------

\begin{frame}{Preprocessing I: Outlier Removal}
	\textbf{Motivation}
	\begin{itemize}
		\item Highly noisy or pathological points:
		\begin{itemize}
			\item are not representative of the underlying geometry or data distribution
			\item distort local geometry,
			\item destabilize spectral embeddings,
			\item bias intrinsic dimension estimates.
		\end{itemize}
		\item For this purpose, we want a \textbf{clean}(i.e. outlier free) subset $\X^{clean} \subseteq \X$.
	\end{itemize}
	\vspace{0.7em}
	\textbf{Recommended approach: Minimum Volume Sets (MVS)}
	\begin{itemize}
		\item Define an $\alpha$-MVS as the smallest set containing at least
		fraction $\alpha$ of the probability mass.
		\item Approximate local density via the same neighbor statistics:
		\begin{itemize}
			\item inverse $k$-NN distance $1/\operatorname{dist}_k(x)$, or
			\item neighbor counts $n_r(x)$ within fixed radii $r$.
		\end{itemize}
		\item Flag points in the lowest-density tail as outliers and remove them.
	\end{itemize}
\end{frame}

\begin{frame}{Outlier Removal in Practice}
	\textbf{Using MVS with neighbor-based scores}
	\begin{itemize}
		\item For each point $x$:
		\begin{itemize}
			\item compute a density proxy (e.g.\ $1/\operatorname{dist}_k(x)$ or $n_r(x)$),
			\item rank all points by this proxy.
		\end{itemize}
		\item Choose a threshold $\alpha \in (0,1)$:
		\begin{itemize}
			\item keep roughly the top $\alpha N$ highest-density points as in-distribution,
			\item discard the remaining $(1-\alpha)N$ as outliers.
		\end{itemize}
		\item Optionally average scores over multiple $k$ / radii for robustness.
	\end{itemize}
	\vspace{0.7em}
	\textbf{Other reasonable options}
	\begin{itemize}
		\item One-Class SVMs, Isolation Forests, KDE-based detectors, etc.
		\item Key design choice: align the outlier detector with the local 
		statistics and geometry used later in the pipeline.
	\end{itemize}
\end{frame}


% ======================================================================
\section{Conclusions and Resources}
% ======================================================================
% ----------------------------------------------------------------------
\begin{frame}{Conclusions \& Take-Home Messages}
	\begin{itemize}
		\item Manifold learning offers powerful tools for Cryo-EM and SBI
		\item Embeddings can reveal intrinsic structure and reduce complexity
		\item SuperMan integrates state-of-the-art statistical diagnostics
	\end{itemize}
\end{frame}

% ----------------------------------------------------------------------
\begin{frame}{Further Reading and Resources}
	\begin{itemize}
		\item Key papers on manifold learning and diffusion maps
		\item References on SBI and applications to Cryo-EM
		\item SuperMan documentation and code repository
	\end{itemize}
\end{frame}

% ======================================================================
\section{Possible Venues}
% ======================================================================
% ----------------------------------------------------------------------
\begin{frame}{Possible Venues for the Tutorial}
	\begin{itemize}
		\item Flatiron Institute — early January
		\item eScience Institute — discuss with D.~Beck
		\item Instats
	\end{itemize}
\end{frame}

% ======================================================================
\section{Ancillary Notes}
% ======================================================================
% ----------------------------------------------------------------------
\begin{frame}{Ancillary Notes}
	\begin{itemize}
		\item \textbf{Friday, Nov.\ 15}: Initial deadline for low-level stabilized code  
		\\(internal note: this was the deadline for the first seminar presentation;
		please communicate if plans change)
		\item \textbf{Monday, Dec.\ 15}: Initial deadline for tutorial materials
	\end{itemize}
\end{frame}

% ======================================================================
\bibliographystyle{ieeetr}
\bibliography{references.bib}
% ======================================================================
\end{document}
