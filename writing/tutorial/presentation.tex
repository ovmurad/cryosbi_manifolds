\documentclass[aspectratio=169]{beamer}
% ======================================================================
% Theme & appearance
% ======================================================================
\usetheme{metropolis}

\metroset{
	progressbar=frametitle,
	numbering=fraction,
	titleformat=smallcaps,
}

\usepackage[T1]{fontenc}
\usepackage[utf8]{inputenc}
\usepackage{lmodern}

% ======================================================================
% Macros & Commands
% ======================================================================
\usepackage{amsmath, amsfonts, amssymb}

% ======================================================================
% Sets
% ======================================================================
\NewDocumentCommand{\R}{o}{\IfValueTF{#1}{\mathbb{R}^{#1}}{\mathbb{R}}}
% ======================================================================
% Comments
% ======================================================================
\NewDocumentCommand{\luke}{m}{{\color{blue}{LE}: #1}}
\NewDocumentCommand{\mmp}{m}{{\color{cyan}{MMP}: #1}}
\NewDocumentCommand{\vlad}{m}{{\color{violet}{OVM}: #1}}

% ======================================================================
% Algos
% ======================================================================
\NewDocumentCommand{\dm}{}{\textsc{DiffMaps} }
\NewDocumentCommand{\tslasso}{}{\textsc{TSLasso} }
\NewDocumentCommand{\ies}{}{\textsc{IES} }
\NewDocumentCommand{\rrelex}{}{\textsc{RRelax} }

% ======================================================================
% Operators
% ======================================================================
\NewDocumentCommand{\dk}{o O{k}}{\IfValueTF{#1}{\mathrm{dist}_{#2}\left(#1\right)}{\mathrm{dist}_{#2}}}
\NewDocumentCommand{\nr}{o O{r}}{\IfValueTF{#1}{\mathrm{n}_{#2}\left(#1\right)}{\mathrm{n}_{#2}}}

% ======================================================================
% Misc
% ======================================================================
\NewDocumentCommand{\codelink}{m}{\href{https://github.com/ovmurad/cryosbi_manifolds}{#1}}
\NewDocumentCommand{\hem}{}{hemagglutinin }
\NewDocumentCommand{\igg}{}{IgG }



% ======================================================================
% Metadata
% ======================================================================
\title{Manifold Learning and Simulation-Based Inference for Cryo-EM}
\author{Luke Evans, Octavian-Vlad Murad, Marina Meilă}
\institute{Flatiron Institute, University of Washington, University of Waterloo}
\date{\today}

% ======================================================================
% Presentation
% ======================================================================
\begin{document}


% ======================================================================
% Title, Contents, Acknowledgement, etc.
% ======================================================================
% ----------------------------------------------------------------------
\begin{frame}
	\titlepage
\end{frame}

% ----------------------------------------------------------------------
\begin{frame}{Outline}
	\tableofcontents
\end{frame}

% ======================================================================
\section{Motivation}
% ======================================================================
% ----------------------------------------------------------------------
\begin{frame}{Motivation}
	\begin{itemize}
		\item Examples of relevant SBI (Simulation-Based Inference) problems
		\item Our main Cryo-EM question
	\end{itemize}
\end{frame}

% ----------------------------------------------------------------------
\begin{frame}{Motivation for Manifold Learning (ML)}
	\begin{itemize}
		\item Many natural data sets are intrinsically low-dimensional
		\item Neural network embeddings are often low-dimensional as well
		\item Therefore: manifold learning can help in many ways (incomplete list):
		\begin{itemize}
			\item estimate dimension
			\item check whether simulated data covers real data
			\item reduce dimension via embedding algorithms
			\item visualize and interpret latent structure
			\item measure distortion
			\item perform SBI in a low-dimensional representation
		\end{itemize}
	\end{itemize}
\end{frame}

% ======================================================================
\section{Basics of ML with SuperMan}
% ======================================================================
% ----------------------------------------------------------------------
\begin{frame}{Basics of ML with SuperMan}
	\begin{itemize}
		\item Brief intro to manifold learning algorithms:
		\begin{itemize}
			\item Diffusion Maps
			\item t-SNE
		\end{itemize}
		\item Why ML is harder than PCA:
		\begin{itemize}
			\item many parameter choices (neighbors $k$, radius $\varepsilon$, metrics, etc.)
		\end{itemize}
		\item Hands-on SuperMan on a toy data set:
		\begin{itemize}
			\item Swiss roll with a hole
		\end{itemize}
	\end{itemize}
\end{frame}

% ----------------------------------------------------------------------
\begin{frame}{Implementation: SuperMan vs.\ scikit-learn}
	\begin{itemize}
		\item Performance:
		\begin{itemize}
			\item SuperMan: fast, real-time or semi-real-time
			\item scikit-learn: slower; may not complete on large data sets
		\end{itemize}
		\item Statistical and geometric features in SuperMan:
		\begin{itemize}
			\item neighbor selection ($k$-NN, radius $\varepsilon$)
			\item Riemannian metric estimation (RMetric)
			\item intrinsic dimension estimation (IES)
			\item other diagnostics and elementary statistics
		\end{itemize}
	\end{itemize}
\end{frame}

% ======================================================================
\section{Pipeline for Cryo-EM}
% ======================================================================
% ----------------------------------------------------------------------
\begin{frame}{The Pipeline for Cryo-EM}
	\begin{itemize}
		\item Subsample for uniformity
		\item Construct neighborhoods
		\item Estimate intrinsic dimension
		\item Assess coverage
		\item RRelax
		\item TSLasso
	\end{itemize}
\end{frame}

% ======================================================================
\section{Conclusions and Resources}
% ======================================================================
% ----------------------------------------------------------------------
\begin{frame}{Conclusions \& Take-Home Messages}
	\begin{itemize}
		\item Manifold learning offers powerful tools for Cryo-EM and SBI
		\item Embeddings can reveal intrinsic structure and reduce complexity
		\item SuperMan integrates state-of-the-art statistical diagnostics
	\end{itemize}
\end{frame}

% ----------------------------------------------------------------------
\begin{frame}{Further Reading and Resources}
	\begin{itemize}
		\item Key papers on manifold learning and diffusion maps
		\item References on SBI and applications to Cryo-EM
		\item SuperMan documentation and code repository
	\end{itemize}
\end{frame}

% ======================================================================
\section{Possible Venues}
% ======================================================================
% ----------------------------------------------------------------------
\begin{frame}{Possible Venues for the Tutorial}
	\begin{itemize}
		\item Flatiron Institute — early January
		\item eScience Institute — discuss with D.~Beck
		\item Instats
	\end{itemize}
\end{frame}

% ======================================================================
\section{Ancillary Notes}
% ======================================================================
% ----------------------------------------------------------------------
\begin{frame}{Ancillary Notes}
	\begin{itemize}
		\item \textbf{Friday, Nov.\ 15}: Initial deadline for low-level stabilized code  
		\\(internal note: this was the deadline for the first seminar presentation;
		please communicate if plans change)
		\item \textbf{Monday, Dec.\ 15}: Initial deadline for tutorial materials
	\end{itemize}
\end{frame}

% ======================================================================
\bibliographystyle{ieeetr}
\bibliography{references.bib}
% ======================================================================
\end{document}
