\documentclass[aspectratio=169]{beamer}
% ======================================================================
% Theme & appearance
% ======================================================================
\usetheme{metropolis}

\metroset{
	progressbar=frametitle,
	numbering=fraction,
	titleformat=smallcaps,
}

\usepackage[T1]{fontenc}
\usepackage[utf8]{inputenc}
\usepackage{lmodern}

% ======================================================================
% Macros & Commands
% ======================================================================
\usepackage{amsmath, amsfonts, amssymb}

% ======================================================================
% Sets
% ======================================================================
\NewDocumentCommand{\R}{o}{\IfValueTF{#1}{\mathbb{R}^{#1}}{\mathbb{R}}}
% ======================================================================
% Comments
% ======================================================================
\NewDocumentCommand{\luke}{m}{{\color{blue}{LE}: #1}}
\NewDocumentCommand{\mmp}{m}{{\color{cyan}{MMP}: #1}}
\NewDocumentCommand{\vlad}{m}{{\color{violet}{OVM}: #1}}

% ======================================================================
% Algos
% ======================================================================
\NewDocumentCommand{\dm}{}{\textsc{DiffMaps} }
\NewDocumentCommand{\tslasso}{}{\textsc{TSLasso} }
\NewDocumentCommand{\ies}{}{\textsc{IES} }
\NewDocumentCommand{\rrelex}{}{\textsc{RRelax} }

% ======================================================================
% Operators
% ======================================================================
\NewDocumentCommand{\dk}{o O{k}}{\IfValueTF{#1}{\mathrm{dist}_{#2}\left(#1\right)}{\mathrm{dist}_{#2}}}
\NewDocumentCommand{\nr}{o O{r}}{\IfValueTF{#1}{\mathrm{n}_{#2}\left(#1\right)}{\mathrm{n}_{#2}}}

% ======================================================================
% Misc
% ======================================================================
\NewDocumentCommand{\codelink}{m}{\href{https://github.com/ovmurad/cryosbi_manifolds}{#1}}
\NewDocumentCommand{\hem}{}{hemagglutinin }
\NewDocumentCommand{\igg}{}{IgG }

%  Environments

\newenvironment{itemize*}{
\begin{itemize}
\setlength{\parskip}{0em}
\setlength{\topparskip}{0em}
}
{\end{itemize}}

\newenvironment{enumerate*}{
\begin{enumerate}
\setlength{\parskip}{0em}
\setlength{\topparskip}{0em}
}
{\end{enumerate}}

\newenvironment{algorithm}[1]{ % argument is name of alg
	\index{Algorithms}{#1}
	\vspace{\vprocskip}
	\centerline{Algorithm {\bf #1}}}
{\vspace{\vprocskip}}


\newcommand{\likecite}[1]{\myblue{\small [#1]}}
\newcommand{\myref}[1]{\myblue{\small (#1)}}



\newcommand{\beq}{\begin{equation}}
\newcommand{\eeq}{\end{equation}}
\newcommand{\beqa}{\begin{eqnarray}}
\newcommand{\eeqa}{\end{eqnarray}}
\newcommand{\beqas}{\begin{eqnarray*}}
\newcommand{\eeqas}{\end{eqnarray*}}

\newcommand{\bit}{\begin{itemize}}
\newcommand{\eit}{\end{itemize}}
\newcommand{\bits}{\begin{itemize*}}
\newcommand{\eits}{\end{itemize*}}
\newcommand{\benum}{\begin{enumerate}}
\newcommand{\eenum}{\end{enumerate}}
\newcommand{\benums}{\begin{enumerate*}}
\newcommand{\eenums}{\end{enumerate*}}
\newcommand{\mybullet}{$\bullet$}

\newcommand{\BlackBox}{\rule{1.5ex}{1.5ex}}  % end of proof

%  math mode commands

\newcommand {\argmax}[2]{\mbox{\raisebox{-1.7ex}{$\stackrel{\textstyle
 {\rm #1}}{\scriptstyle #2}$}}\,}
\newcommand{\nchoosem}[2]{\left(\!\!\!\begin{array}{c}#1\\#2\end{array}\!\!\!\right)}
\newcommand{\fracpartial}[2]{\frac{\partial #1}{\partial  #2}}

\newcommand{\bigOO}{{\cal O}}
\newcommand{\dataset}{{\cal D}}
\newcommand{\model}{{\cal M}}

\newcommand{\KLdiv}[2]{KL(#1\,||\,#2)}

\newcommand{\rrr}{{\mathbb R}}
\newcommand{\neigh}{\operatornam{neigh}}
\DeclareMathOperator{\trace}{trace}
\DeclareMathOperator{\diag}{diag}

\newcommand{\onevector}{{\mathbf 1}}
\newcommand{\bbone}[1]{{1}_{[#1]}}


\newlength{\picwi}
\newlength{\tttwi}
\setlength{\tttwi}{0.9\textwidth}

\newlength{\backitem}
\setlength{\backitem}{-2.5em}
\newcommand{\backskip}{\hspace{-2.5em}} % how much to skip back for an empty item?
\newlength{\colwi}  % column width

\definecolor{mynavy}{RGB}{0,85,166} 

\definecolor{myblue}{RGB}{50,132,191}  %{uclablue}
\definecolor{myred}{rgb}{0.74,0.1,0.05}
\definecolor{mygreen}{rgb}{0,0.52,0.32}
\definecolor{myyellow}{RGB}{255,232,0}%uclagold{rgb}{0.96,0.92,0.13}
\definecolor{myorange}{RGB}{254,187,54}%uclagolddark {rgb}{0.7,0.41,0.1}
\definecolor{mypurple}{RGB}{130,55,255}%{uclamov}%{rgb}{0.51,0.02,.8}
\definecolor{mygray}{rgb}{0.6,0.6,0.6}

\newcommand{\myblue}[1]{\textcolor{myblue}{#1}}
\newcommand{\myred}[1]{\textcolor{myred}{#1}}
\newcommand{\mygreen}[1]{\textcolor{mygreen}{#1}}
\newcommand{\myorange}[1]{\textcolor{myorange}{#1}}
\newcommand{\myyellow}[1]{\textcolor{myellow}{#1}}
\newcommand{\mypurple}[1]{\textcolor{mypurple}{#1}}
\newcommand{\mygray}[1]{\textcolor{mygray}{#1}}

\newcommand{\mydef}[1]{\myred{ {#1}}}
\newcommand{\myemph}[1]{\mygreen{ {#1}}}
  % temp, until they are merged

\graphicspath{{../figures/}{../figures/figures-MMP/}}

% ======================================================================
% Metadata
% ======================================================================
\title{Manifold Learning and Simulation-Based Inference for Cryo-EM}
\author{Octavian-Vlad Murad, Luke Evans, Marina Meilă}
\institute{University of Washington, Flatiron Institute, University of Waterloo}
\date{\today}

% ======================================================================
% Presentation
% ======================================================================
\begin{document}


% ======================================================================
% Title, Contents, Acknowledgement, etc.
% ======================================================================
% ----------------------------------------------------------------------
\begin{frame}
	\titlepage
\end{frame}

% ----------------------------------------------------------------------
\begin{frame}{Outline}
	\tableofcontents
\end{frame}

% ======================================================================
\section{Motivation}
% ======================================================================
% ----------------------------------------------------------------------
\begin{frame}{Motivation}
	\begin{itemize}
		\item Examples of relevant SBI (Simulation-Based Inference) problems
		\item Our main Cryo-EM question
	\end{itemize}
\end{frame}

% ----------------------------------------------------------------------
\begin{frame}{Motivation for Manifold Learning (ML)}
	\begin{itemize}
		\item Many natural data sets are intrinsically low-dimensional
		\item Neural network embeddings are often low-dimensional as well
		\item Therefore: manifold learning can help in many ways (incomplete list):
		\begin{itemize}
			\item estimate dimension
			\item check whether simulated data covers real data
			\item reduce dimension via embedding algorithms
			\item visualize and interpret latent structure
			\item measure distortion
			\item perform SBI in a low-dimensional representation
		\end{itemize}
	\end{itemize}
\end{frame}
% ----------------------------------------------------------------------
%% Added slides from my MMP's tutorial
% ----------------------------------------------------------------------
%------------------------------------------------------------------------
\begin{frame}\frametitle{Spectra of galaxies measured by the \mygreen{Sloan Digital Sky Survey (SDSS)}}
%\begin{frame}\frametitle{Spectra of galaxies measured by the \mygreen{Sloan Digital Sky Survey (SDSS)}}
\begin{columns}
\begin{column}{0.3\tttwi}
\setlength{\picwi}{\textwidth}
\centerline{\mygray{\tiny www.sdss.org}}
\includegraphics[width=\picwi]{pie_boss_z0-710.jpg}
\\
\centerline{\mygray{\tiny www.sdss.org}}
\includegraphics[width=\picwi]{specById_asp.png}
\end{column}
%
\begin{column}{0.7\tttwi}
\bit
%\item Spectra of galaxies measured by the \mygreen{Sloan Digital Sky Survey (SDSS)}
\item {\small Preprocessed by Jacob VanderPlas and Grace Telford}
\item $n=675,000$ spectra $\times\,D=3750$ dimensions
\eit
\includegraphics[width=0.6\tttwi]{pcEmbed_SFR-trim.png}
\centerline{\mygray{\tiny embedding by James McQueen}}

\end{column}
\end{columns}
\end{frame}
%------------------------------------------------------------------------
\begin{frame}\frametitle{Molecular configurations}
\begin{columns}
\begin{column}{0.35\tttwi}
\setlength{\picwi}{\textwidth}
\centerline{{\small aspirin molecule}}
\includegraphics[width=\picwi]{aspirin.png}
\\
%\includegraphics[width=\picwi]{aspirin-R33-82.png}
\end{column}
\begin{column}{0.7\tttwi}
\bit
\item Data from \mygreen{Molecular Dynamics (MD)} simulations of small molecules by \likecite{Chmiela et al. 2016}
\item $n\approx 200,000$ configurations $\times\,D\sim 20-60$ dimensions
\item[]
\eit
\includegraphics[width=0.6\tttwi]{fig-aspirin-cc.png}
\end{column}
\end{columns}

\end{frame}
%------------------------------------------------------------------------
\begin{frame}<beamer:0|handout:0>\frametitle{ When to do (non-linear) dimension reduction}
%\begin{frame}\frametitle{ When to do (non-linear) dimension reduction}
\setlength{\picwi}{0.6\textwidth}
\centerline{\includegraphics[width=\picwi]{Figures/IsomapFaces.png}}
\bit
\item high-dimensional \myblue{data $p \in \rrr^D,\,D=64\times 64$}
\item can be described by a small number $d$ of continuous parameters 
\item Usually, large sample size \myblue{n}
\eit
\end{frame}
%------------------------------------------------------------------------
%\begin{frame}<beamer:0|handout:0>\frametitle{ When to do (non-linear) dimension reduction}
\begin{frame}\frametitle{ When to do (non-linear) dimension reduction}

%\setlength{\picwi}{0.6\textwidth}
%\centerline{\includegraphics[width=\picwi]{Figures/IsomapFaces.png}}

%
\begin{center}
\setlength{\picwi}{0.3\tttwi}
\begin{tabular}{ccc}
  {\tiny HR diagram} & {\tiny aspirin MD simulation} & {\tiny SDSS galaxy spectra} \\
 \includegraphics[width=0.7\picwi]{Figures/HRDiagram.png}&
\hspace{-.3em}\includegraphics[width=1.\picwi]{fig-aspirin-cc.png}
&
\includegraphics[width=1.\picwi]{pcEmbed_SFR-trim.png}\\
\end{tabular}
\end{center}

\only<1>{
  \bit
\item high-dimensional %\myblue{data $p \in \rrr^D,\,D=64\times 64$}
\item can be described by a small number $d$ of continuous parameters 
\item Usually, large sample size \myblue{n}
\eit
}%end only 1
\only<2>{
Why?
 \bit 
\item To save space and computation
  \bit
  \item $n\times D$ data matrix $\rightarrow\;n\times m$, $m\ll D$
      \eit
\item To use it afterwards in (\myemph{prediction}) tasks
\item To \myemph{understand} the data better
  \bit
  \item preserve large scale features, suppress fine scale features
  \eit
  \eit
  }%end only 2
\end{frame}
%------------------------------------------------------------------------
%\begin{frame}<beamer:0|handout:0>\frametitle{Manifold Learning (ML) for the physical sciences}
\begin{frame}\frametitle{Manifold Learning (ML) for the physical sciences}
  \bit
\item big, high-dimensional data
\item data, physics supports manifold models
\item understanding \& prediction equally important
\item[]
%\item \myemph{Simple assumption} data i.i.d. from manifold $\M$ (no noise)
  \pause
\eit
\mydef{Challenges} for ML algorithms
  \bit
\item scalable  \myblue{\tt mega}\myorange{man}
  \pause
  ML package \likecite{McQueen et al JMLR 2015}  \myref{\checkmark}
\pause
\item \mygreen{find ``something new, trustworthy, reproducible, interpretable''}
  \pause
\item remove algorithmic artefacts \mygreen{(replace grad student)}
\item data-driven parameter selection  \mygreen{(replace grad student)}
\item validation on mathematical/statistical grounds as much as possible \mygreen{(replace experimental validation)}
\item use \myred{domain knowledge} \mygreen{(not domain expert)}
\eit
\end{frame}
%------------------------------------------------------------------------
%\begin{frame}\frametitle{Brief intro to manifold learning algorithms} 
\begin{frame}<beamer:0|handout:0>\frametitle{Brief intro to manifold learning algorithms} 
\setlength{\picwi}{0.3\textwidth}
\bit
\item {\bf Input} Data $p_1,\ldots p_n$, embedding dimension \myblue{$m$}, neighborhood scale parameter \myblue{$\epsilon$}
\only<2->{\item \mygreen{Construct neighborhood graph} $p,p'$ neighbors iff {\small $||p-{p'}||^2\leq \epsilon$}}
\only<3->{\item Construct a \myred{$n\times n$ matrix}:
 its leading eigenvectors are the \myblue{coordinates $\phi(p_{1:n})$}}
\\
\only<1-3>{\includegraphics[width=.7\picwi,height=.5\picwi,viewport=100 100 485 360,clip]{../dominique-epsilon/Figures/pretty-hourglass-sample-n2000.png}}
\hspace{1em}
\only<2-3>{ \includegraphics[width=.8\picwi]{../dominique-epsilon/Figures/fig_hourglass2D-n500-snoise0_001-graph-eps0_3-tol0_25.png}}
\hspace{1em}
\only<3>{\includegraphics[width=.7\picwi]{../dominique-epsilon/Figures/pretty-hourglass-laplacian.png}}\\
\only<1-3>{$p_1,\ldots p_n\,\subset\,\rrr^D$}
\eit
 
\end{frame}
%------------------------------------------------------------------------
%------------------------------------------------------------------------
\begin{frame}<beamer:0|handout:0>\frametitle{Brief intro to manifold learning algorithms} % WITHOUT Isomap **** with Isomap is next slide. Remove <...> to make animation work
\setlength{\picwi}{0.3\textwidth}
\bit
\item {\bf Input} Data $p_1,\ldots p_n$, embedding dimension \myblue{$m$}, neighborhood scale parameter \myblue{$\epsilon$}
\only<2->{\item \mygreen{Construct neighborhood graph} $p,p'$ neighbors iff {\small $||p-{p'}||^2\leq \epsilon$}}
\only<3->{\item Construct a \myred{$n\times n$ matrix}:
 its leading eigenvectors are the \myblue{coordinates $\phi(p_{1:n})$}}
\\
\only<1-3>{\includegraphics[width=.7\picwi,height=.5\picwi,viewport=100 100 485 360,clip]{../dominique-epsilon/Figures/pretty-hourglass-sample-n2000.png}}
\hspace{1em}
\only<2-3>{ \includegraphics[width=.8\picwi]{../dominique-epsilon/Figures/fig_hourglass2D-n500-snoise0_001-graph-eps0_3-tol0_25.png}}
\hspace{1em}
\only<3>{\includegraphics[width=.7\picwi]{../dominique-epsilon/Figures/pretty-hourglass-laplacian.png}}\\
\only<1-3>{$p_1,\ldots p_n\,\subset\,\rrr^D$}

\item[]
\only<4>{
\item[]{\sc Laplacian Eigenmaps/Diffusion Maps} \likecite{Belkin,Niyogi 02,Nadler et al 05}
\item  Construct similarity matrix
\[S=[S_{pp'}]_{p,p'\in \dataset}\quad \text{with}\quad
S_{pp'}=e^{-\frac{1}{\epsilon}||p-p'||^2}\quad\text{iff $p,p'$ neighbors}
\]
\item Construct \myred{Laplacian matrix} $L=I-T^{-1}S$ with $T={\rm diag}(S{\mathbf 1})$
\item Calculate  $\phi^{1\ldots m}\,=\,$ eigenvectors of $L$ {\small (smallest eigenvalues)}
\item coordinates of $p\in \dataset$ are $(\phi^1(p),\,\ldots\,\phi^m(p))$\\
\item[]}
\eit
 
\end{frame}
%------------------------------------------------------------------------
\begin{frame}<beamer:0|handout:0>\frametitle{Brief intro to manifold learning algorithms}  % WITH Isomap  <beamer:0|handout:0> must be deleted for animation to work
%\begin{frame}\frametitle{Brief intro to manifold learning algorithms} 
\setlength{\picwi}{0.3\textwidth}
\bit
\item {\bf Input} Data $p_1,\ldots p_n$, embedding dimension \myblue{$m$}, neighborhood scale parameter \myblue{$\epsilon$}
\only<2->{\item \mygreen{Construct neighborhood graph} $p,p'$ neighbors iff {\small $||p-{p'}||^2\leq \epsilon$}}
\hfill{\myemph{ALL ALGORITHMS}}
\only<3>{\item Construct a \myred{$n\times n$ matrix}:
 its leading eigenvectors are the \myblue{coordinates $\phi(p_{1:n})$}}
\\
\only<1-3>{\includegraphics[width=.7\picwi,height=.5\picwi,viewport=100 100 485 360,clip]{pretty-hourglass-sample-n2000.png}}
\hspace{1em}
\only<2-3>{ \includegraphics[width=.8\picwi]{fig_hourglass2D-n500-snoise0_001-graph-eps0_3-tol0_25.png}}
\hspace{1em}
\only<3>{\includegraphics[width=.7\picwi]{pretty-hourglass-laplacian.png}}\\
\only<1-3>{$p_1,\ldots p_n\,\subset\,\rrr^D$}

\item[]
\only<4>{
\item[]{\sc Laplacian Eigenmaps/Diffusion Maps} \likecite{Belkin,Niyogi 02,Nadler et al 05}
\item  Construct similarity matrix
\[S=[S_{pp'}]_{p,p'\in \dataset}\quad \text{with}\quad
S_{pp'}=e^{-\frac{1}{\epsilon}||p-p'||^2}\quad\text{iff $p,p'$ neighbors}
\]
\item Construct \myred{Laplacian matrix} $L=I-T^{-1}S$ with $T={\rm diag}(S{\mathbf 1})$
\item Calculate  $\phi^{1\ldots m}\,=\,$ eigenvectors of $L$ {\small (smallest eigenvalues)}
\item coordinates of $p\in \dataset$ are $(\phi^1(p),\,\ldots\,\phi^m(p))$\\
\item[]}
\only<5>{  % should be <5>
\item[]{\sc Isomap} \likecite{Tennenbaum, deSilva \& Langford 00}
\item Find all shortest paths in neighborhood graph, construct \myred{matrix of distances}
\[M=[\text{distance}^2_{pp'}]\]
\item use $M$ and \myblue{Multi-Dimensional Scaling (MDS)} to obtain $m$ dimensional coordinates for $p\in \dataset$}
\eit
 
\end{frame}
%------------------------------------------------------------------------
\begin{frame}<beamer:0|handout:0>\frametitle{Isomap vs. Diffusion Maps}
%\begin{frame}\frametitle{Isomap vs. Diffusion Maps}

\begin{columns}
\begin{column}{0.5\tttwi}
\includegraphics[trim= 1.5cm 1.5cm 1.5cm 1.5cm,clip=true,width=0.46\textwidth]{PARISDominique/Figures/IsomapFaces.png}  

\mydef{Isomap}
\bit
\item Preserves geodesic distances
  \bit
  \item but only sometimes \likecite{}
  \eit
\item Computes all-pairs shortest paths $\bigOO(n^3)$
\item Stores/processes \myemph{dense} matrix 
\eit
\end{column}
\begin{column}{0.5\tttwi}
\includegraphics[trim= 1.5cm 1.5cm 1.5cm 1.5cm,clip=true,width=0.46\textwidth]{PARISDominique/Figures/EigenmapFaces.png} 

\mydef{DiffusionMap}
\bit
\item Distorts geodesic distances
\item Computes only distances to nearest neighbors $\bigOO(n^{1+\epsilon})$
\item Stores/processes \myemph{sparse} matrix 
\eit
\end{column}
\end{columns}
\vfill
\bit
\item t-SNE, UMAP visualization algorithms, heuristic
\eit
\end{frame}
%------------------------------------------------------------------------
\begin{frame}<beamer:0|handout:0>\frametitle{}
%\begin{frame}\frametitle{}
A toy example (the ``Swiss Roll'' with a hole)

\vspace{2em}
\setlength{\picwi}{0.55\textwidth}
\begin{tabular}{cc}
points in $D\geq 3$ dimensions & same points reparametrized in 2D \\ 
\includegraphics[width=0.8\picwi]{Figures/mani-swisshole-1000-k8-isomap-6-data.png} 
&
\hspace{-2.5em}\includegraphics[width=1.2\picwi,height=0.3\picwi]{Figures/mani-swisshole-1000-k8-ltsa-6-embe.png}
\\
\mydef{Input} & \mydef{Desired output}
\end{tabular}
\end{frame}
%------------------------------------------------------------------------
\begin{frame}<beamer:0|handout:0>\frametitle{Embedding in 2 dimensions by different manifold learning algorithms}
%\begin{frame}\frametitle{Embedding in 2 dimensions by different manifold learning algorithms}
\hspace{5.7em}\mydef{Input}\\
\setlength{\picwi}{0.9\textwidth}
\begin{center}
\includegraphics[width=\picwi]{Figures/fig-mani-swisshole.png}
\\\mygray{\tiny Figure by Todd Wittman}
\end{center}
\end{frame}
%------------------------------------------------------------------------
\begin{frame}<beamer:0|handout:0>\frametitle{Embedding in 2 dimensions by different manifold learning algorithms}
%\begin{frame}\frametitle{Embedding in 2 dimensions by different manifold learning algorithms}
\setlength{\picwi}{0.3\textwidth}

\begin{small}
\begin{columns}
\begin{column}{\picwi}
Original data\\{\small  (Swiss Roll with hole)}\\
\includegraphics[width=\picwi]{Figures/mani-swisshole-1000-k8-isomap-6-data.png}\\
{\small Hessian Eigenmaps (HE)}\\
\includegraphics[width=\picwi]{Figures/mani-swisshole-1000-k12-he-6-embe.png}\\
\end{column}
\begin{column}{\picwi}
{\small Laplacian Eigenmaps (LE)}\\
\includegraphics[width=\picwi]{Figures/mani-swisshole-1000-k8-le-embe.png}\\
%\includegraphics[width=\picwi]{Figures/mani-swisshole-1000-k8-le-6-embe.png}\\
% chilot
{\small Local Linear Embedding (LLE)}\\
\includegraphics[width=\picwi]{Figures/mani-swisshole-1000-k8-lle-6-embe.png}\\
\end{column}
\begin{column}{\picwi}
{\small Isomap}\\
\includegraphics[width=\picwi]{Figures/mani-swisshole-1000-k8-isomap-6-embe.png}\\
{\small Local Tangent Space Alignment (LTSA)}\\
\includegraphics[width=\picwi]{Figures/mani-swisshole-1000-k8-ltsa-6-embe.png}\\
\end{column}
\end{columns}
\end{small}
\end{frame}
%------------------------------------------------------------------------
%\begin{frame}<beamer:0|handout:0>\frametitle{How to evaluate the results objectively?}
\begin{frame}<beamer:0|handout:0>\frametitle{How to evaluate the results objectively?}
\bit
\item Many algorithms exist
\myblue{Isomap, Laplacian Eigenmaps (LE), Diffusion Maps (DM), Hessian Eigenmaps (HE), Local Linear Embedding (LLE), Latent Tangent Space Alignment (LTSA)}
\item Each algorithm gives a different embedding of the same data
\eit

\setlength{\picwi}{0.15\textwidth}
\begin{small}
\begin{columns}
\begin{column}{\picwi}
Original \\
\includegraphics[width=\picwi]{Figures/mani-swisshole-1000-k8-isomap-6-data.png}\\
{\tiny Hessian Eigenmaps (HE)}\\
\includegraphics[width=\picwi]{Figures/mani-swisshole-1000-k12-he-6-embe.png}\\
\end{column}
\begin{column}{\picwi}
{\tiny LE}\\
\includegraphics[width=\picwi]{Figures/mani-swisshole-1000-k8-le-embe.png}\\
%\includegraphics[width=\picwi]{Figures/mani-swisshole-1000-k8-le-6-embe.png}\\
% chilot
{\tiny LLE}\\
\includegraphics[width=\picwi]{Figures/mani-swisshole-1000-k8-lle-6-embe.png}\\
\end{column}
\begin{column}{\picwi}
{\tiny Isomap}\\
\includegraphics[width=\picwi]{Figures/mani-swisshole-1000-k8-isomap-6-embe.png}\\
{\tiny LTSA}\\
\includegraphics[width=\picwi]{Figures/mani-swisshole-1000-k8-ltsa-6-embe.png}\\
\end{column}
\begin{column}{0.7\textwidth}
\bit
\item which of these embedding are ``correct''?
\item if several ``correct'', how do we reconcile them?
\item if not ``correct'', what failed? 
\eit
\end{column}
\end{columns}
\end{small}
\end{frame}
%------------------------------------------------------------------------
\begin{frame}<beamer:0|handout:0>\frametitle{How to evaluate the results objectively?}
%\begin{frame}\frametitle{How to evaluate the results objectively?}

\setlength{\picwi}{0.45\textwidth}
\begin{small}
\begin{columns}
\begin{column}{\picwi}
\only<1>{\includegraphics[width=\picwi]{Figures/fig-mani-swisshole.png}}
\only<2>{%\includegraphics[width=0.5\picwi]{Figures/HRDiagram.png}
\includegraphics[width=\picwi]{Figures/jvdp-spectra-data.png}
\\
\mygray{\tiny Spectrum of a galaxy. Source SDSS, Jake VanderPlas}
}
\end{column}
\begin{column}{0.7\textwidth}
\bit
\item which of these embedding are ``correct''?
\item if several ``correct'', how do we reconcile them?
\item if not ``correct'', what failed?
\only<2>{\item what if I have real data?} 
\eit
\end{column}
\end{columns}
\end{small}
\only<1>{\tiny 
Algorithms \myblue{Multidimensional Scaling (MDS), Principal Components (PCA), Isomap, Locally Linear Embedding (LLE),  Hessian Eigenmaps (HE), Laplacian Eigenmaps (LE), Diffusion Maps (DM)}
}
\end{frame}
%------------------------------------------------------------------------

% ======================================================================
\section{Basics of ML with SuperMan}
% ======================================================================
% ----------------------------------------------------------------------
\begin{frame}{Basics of ML with SuperMan}
	\begin{itemize}
		\item Brief intro to manifold learning algorithms:
		\begin{itemize}
			\item Diffusion Maps
			\item t-SNE
		\end{itemize}
		\item Why ML is harder than PCA:
		\begin{itemize}
			\item many parameter choices (neighbors $k$, radius $\varepsilon$, metrics, etc.)
		\end{itemize}
		\item Hands-on SuperMan on a toy data set:
		\begin{itemize}
			\item Swiss roll with a hole
		\end{itemize}
	\end{itemize}
\end{frame}

% ----------------------------------------------------------------------
%\begin{frame}\frametitle{PCA -- Linear dimension reduction}
%\end{frame}
%------------------------------------------------------------------------
\begin{frame}\frametitle{A toy example (the ``Swiss Roll'' with a hole)}
\setlength{\picwi}{0.5\tttwi}
\begin{columns}
\begin{column}{0.4\tttwi}
  \centerline{\mydef{Input}}
  points in $\dhigh\geq 3$ dimensions 
  \includegraphics[width=0.8\picwi]{Figures/mani-swisshole-1000-k8-isomap-6-data.png}
\end{column}
\begin{column}{0.6\tttwi}
  \centerline{\mydef{Desired output}}
    same points reparametrized in 2D 
\includegraphics[width=1.2\picwi,height=0.3\picwi]{Figures/mani-swisshole-1000-k8-ltsa-6-embe.png}
\pause
Linear dimension reduction fails\\
\includegraphics[width=0.4\picwi,height=0.6\picwi]{Figures/fig-mani-swisshole-pcacut.png}
\hspace{1em}\includegraphics[width=0.4\picwi,height=0.6\picwi]{Figures/fig-mani-swisshole-mdscut.png}
\end{column}
\end{columns}

\end{frame}
%------------------------------------------------------------------------
\begin{frame}<beamer:0|handout:0>\frametitle{Neighborhood graphs}
%\begin{frame}\frametitle{Neighborhood graphs}

\bit
\item All ML algorithms start with a \mydef{neighborhood graph} over the data points
  \bit
\item  $\neigh_i$ denotes the neighbors of $\xi_i$, and $k_i=|\neigh_i|$. 
\item  $\Xi_i=[\xi_{i'}]_{i'\in\neigh_i}\in \rrr^{\dhigh\times k_i}$ contains the coordinates of $\xi_i$'s neighbors
  \eit
\item In the \mydef{radius-neighbor} graph, the neighbors of $\xi_i$ are the points within distance $r$ from $\xi_i$, i.e. in the ball $B_r(\xi_i)$.
\item In the \mydef{k-nearest-neighbor (k-nn)} graph, they are the $k$ nearest-neighbors of $\xi_i$. 


\item[]
\item k-nn graph has many computational advantages
  \bit
\item constant degree $k$ (or $k-1$) 
\item connected for any $k>1$
\item more software available
\item[]
  \item but much more difficult to use for \myemph{consistent} estimation of manifolds (see later, and \cite{}) 
    \eit
   \eit
   \setlength{\picwi}{0.3\textwidth}

   \begin{tabular}{ccc}
\includegraphics[width=.7\picwi,height=.5\picwi,viewport=100 100 485 360,clip]{pretty-hourglass-sample-n2000.png}&
\hspace{1em}
\includegraphics[width=.8\picwi]{fig_hourglass2D-n500-snoise0_001-graph-eps0_3-tol0_25.png}&
\hspace{1em}
\includegraphics[width=.7\picwi]{pretty-hourglass-sample-n2000.png}&
\hspace{1em}
\includegraphics[width=.8\picwi]{fig_hourglass2D-n500-snoise0_001-graph-eps0_3-tol0_25.png}&
\hspace{1em}
\includegraphics[width=.7\picwi]{pretty-hourglass-laplacian.png}\\
data $\xi_1,\ldots \xi_n\,\subset\,\rrr^D$
&
neighborhood graph
&
$A$ (sparse) matrix of\\
&&distances between neighbors\\
\end{tabular}
\end{frame}
%------------------------------------------------------------------------
\subsection{E-vector based embedding algorithms}
%------------------------------------------------------------------------
\begin{frame}<beamer:0|handout:0>\frametitle{Embedding algorithms}
%\begin{frame}\frametitle{Embedding algorithms}

Diffusion Maps/Laplacian Eigenmaps, Isomap, LTSA, MVU, Hessian Eigenmaps, SketchMap \likecite{}\ldots
  
\bit
\item Map $\dataset$ to $\rrr^m$ where $m\geq d$ (global coordinates)
\item Can also map a local neighborhood $U\subseteq \dataset$ to $\rrr^d$ (local, intrinsic coordinates)
 \item[]
  \item[]\mydef{Input}
  \item   embedding dimension $m$
  \item  neighborhood radius/kernel width {$\epsilon$}
    \bit
  \item usually radius $r\approx 3\times \epsilon$ 
    \eit
  \item  neighborhood graph
    
  \item[] $\{\neigh_i,\,\Xi_i,\text{ for }i=1:n\}$
  \item[] $A=[\|\xi_i-\xi_j\|]_{i,j=1}^n$ distance matrix, with $A_{ij}=\infty$ if $i\not\in\neigh_j$ 
\eit

\end{frame}

%------------------------------------------------------------------------
%\begin{frame}\frametitle{The Isomap algorithm}
\begin{frame}<beamer:0|handout:0>\frametitle{The Isomap algorithm}
\begin{block}{Isomap Algorithm \likecite{Tennenbaum, deSilva \& Langford 00}}
  \benum
\item[]{\bf Input} $A$, dimension $d$ 
\item Find all shortest path distances in neighborhood graph
\item[] if $A_{ij}=\infty$, then $A_{ij}\,\gets$ graph distance between $i,j$
\item Construct \myred{matrix of squared distances}
\[M=[(A_{ij})^2]\]
  \pause
\item use \myblue{Multi-Dimensional Scaling}  MDS$(M,d)$ to obtain $d$ dimensional coordinates $Y$ for $\dataset$
\eenum
\end{block}

\bit
\item Works also for $m>d$
\eit
\end{frame}
%------------------------------------------------------------------------

\begin{frame}\frametitle{The Diffusion Maps (DM)/ Laplacian Eigenmaps (LE) Algorithm}
\begin{block}{Diffusion Maps Algorithm}
\benum
\item[] {\bf Input} distance matrix $A\in \rrr^{n\times n}$ , bandwidth $\epsilon$, embedding dimension $m$
\item Compute Laplacian $L\in \rrr^{n\times n}$
\item Compute eigenvectors of $L$ for \myemph{smallest $m+1$ eigenvalues} $[\phi_0\,\phi_1\,\ldots \phi_m]\in\rrr^{n\times m}$
  \bit
  \item $\phi_0$ is constant and not informative
%  \item These are the \myemph{slow modes} of the system
  \eit
\item[] The \mydef{embedding coordinates} of $p_{i}$ are $(\phi_{i1},\ldots
  \phi_{im})$
\eenum
\end{block}
\setlength{\picwi}{0.3\textwidth}
%\hfill\includegraphics[width=\picwi]{Figures-aspirin/ethanol-tau10-noaxes.png}
\end{frame}
%------------------------------------------------------------------------
%\begin{frame}\frametitle{The (renormalized) Laplacian}
\begin{frame}<beamer:0|handout:0>\frametitle{The Laplacian}
\begin{block}{Laplacian}
\benum
\item[] Input distance matris $A\in\rrr^{n\times n}$, \mydef{bandwidth} ${\epsilon}$
\item Compute \mydef{similarity matrix} $S_{ij}=\exp\left(-\frac{A_{ij}^2}{\epsilon^2}\right)=\kappa(A_{ij}/\epsilon)$
  \pause
\item Normalize columns $\quad\quad d_j=\sum_{i=1}^n S_{ij}$, $\quad\tilde{L}_{ij}=S_{ij}/{d_j}$
\item Normalize rows $\quad\quad\quad d'_i=\sum_{j=1}^n \tilde{L}_{ij}$, $\quad P_{ij}=\tilde{L}_{ij}/d_i'$
  \pause
\item $L=\frac{1}{\epsilon^2}(I-P)$
\item{Output} $L$, $d'_{i}/d_i$
\eenum
\end{block}
\pause
\bit
\item Laplacian $L$ central to understanding the manifold geometry
\item $\lim_{n\rightarrow \infty}L\,=\,\Delta_\M$ \likecite{Coifman,Lafon 2006} 
\item Renormalization trick cancels effects of (non-uniform) sampling density \likecite{Coifman,Lafon 2006}  
\item[]
 \pause
\item[] Other Laplacians
\item $L^{un}=\diag\{d_{1:n}\}-A$ \hfill\myemph{unnormalized} Laplacian
\item $L^{rw}=I-\diag\{d_{1:n}\}^{-1}A$ \hfill\myemph{random walk} Laplacian
\item $L^{n}=I-\diag\{d_{1:n}\}^{-1/2}A\diag\{d_{1:n}\}^{-1/2}$ \hfill\myemph{normalized} Laplacian
  \eit
\note{\begin{myproblem}{Renormalized Laplacian}
{\bf a.} Show that $L\bbone{}=0$ for the renormalized Laplacian. Hence $L$ always has a 0 e-value. 
  \end{myproblem}
%-------------------------------------
  \begin{myproblem}[Unnormalized Laplacian]
  Let $L^{un}=D-A$ be the {\em unnormalized Laplacian} of graph defined by $A$. Prove that $x^TL^{un}x=\sum_{(i,j)\in{\cal E}} (x_i-x_j)^2$ for any $x\in\rrr^n$.
  \end{myproblem}
  %-------------------------------------
  \begin{myproblem}[Double Normalization Laplacian]
A more standard presentation of the Re-normalized Laplacian is this:
\benum
\item Compute similarity matrix $S$
\item First normalization $d_i=\sum_{j=1}^n S_{ij}$, $\tilde{L}_{ij}=S_{ij}/{d_id_j}$ (symmetric matrix)
\item Second normalization $d'_i=\sum_{j=1}^n \tilde{L}_{ij}$, $P_{ij}=\tilde{L}_{ij}/d_i'$ (asymmetric)
\item $L=\frac{1}{\epsilon^2}(I-P)$
  \eenum
 Show that this $L$ is the same as in the algorithm on the previous page.
    \end{myproblem}
%-------------------------------------
}% end note
\end{frame}
%------------------------------------------------------------------------
\begin{frame}<beamer:0|handout:0>\frametitle{Isomap vs. Diffusion Maps}
%\begin{frame}\frametitle{Isomap vs. Diffusion Maps}

\begin{columns}
\begin{column}{0.5\tttwi}
\includegraphics[trim= 1.5cm 1.5cm 1.5cm 1.5cm,clip=true,width=0.46\textwidth]{PARISDominique/Figures/IsomapFaces.png}  

\mydef{Isomap}
\bit
\item Preserves geodesic distances
  \bit
  \item but only when $\M$ is \myemph{flat} and ``data'' convex
  \eit
\item Computes all-pairs shortest paths $\bigOO(n^3)$
\item Stores/processes \myemph{dense} matrix 
\eit
\end{column}
\begin{column}{0.5\tttwi}
\includegraphics[trim= 1.5cm 1.5cm 1.5cm 1.5cm,clip=true,width=0.46\textwidth]{PARISDominique/Figures/EigenmapFaces.png} 

\mydef{DiffusionMap}
\bit
\item Distorts geodesic distances
\item Computes only distances to nearest neighbors $\bigOO(n^{1+\epsilon})$  % or sqrt{\epsilon}??
\item Stores/processes \myemph{sparse} matrix 
\eit
\end{column}
\end{columns}
\vfill
\bit
\item t-SNE, UMAP visualization algorithms
\eit
\end{frame}
%\end{document} %%%%%%%%%%%%%%
%------------------------------------------------------------------------
\subsection{Repulsion-based algorithms}
%------------------------------------------------------------------------
%\begin{frame}\frametitle{Repulsion-based (heuristic) algorithms}
\begin{frame}<beamer:0|handout:0>\frametitle{Repulsion-based (heuristic) algorithms}
  
\begin{block}{t-Stochastic Neighbor Embedding (t-SNE)}
  \benum
\item[Input] similarity matrix  $S$, embedding dimension $s$
\item[Init] choose embedding points $y_{1:n}\in\rrr^s$ at random
\item\label{step:q} $S_{ii}\gets 0$, normalize rows $d_i=\sum_j S_{ij}$, $P_{ij}=S_{ij}/d_i$
\item symmetrize $P=\frac{1}{2n}(P+P^T)$ \myemph{$P$ is distribution over pairs of neighbors $(i,j)$}
\item  $\tilde{S}_{ij}=\tilde{\kappa}(\|y_i-y_j\|)$\myemph{compute similarity in output space}
\item[]where $\tilde{\kappa}(z)=\frac{1}{1+z^2}$ the \myemph{Cauchy (Student t with 1 degree of freedom)}
\item Define distribution $Q$ with $Q_{ij}\propto S_{ij}$
\item Change $y_{i:n}$ to decrease the \mydef{Kullbach-Leibler divergence} $KL(P||Q)=\sum_{i,j}P_{ij}\ln \frac{P_{ij}}{Q_{ij}}$ (by gradient descent) and repeat from step \ref{step:q}
  \eenum
\end{block}

\pause
\bit
\item empirically useful for visualizing clusters (repulsion encourages cluster formation)
\item non-deterministic, more parameters
\eit
 
\end{frame}
%------------------------------------------------------------------------
\begin{frame}<beamer:0|handout:0>\frametitle{UMAP: Uniform Manifold Approximation and Projection}
%\begin{frame}\frametitle{UMAP: Uniform Manifold Approximation and Projection \likecite{McInnes, Healy, Melville,2018}}
  \setlength{\picwi}{0.25\textwidth}
\begin{columns}
\begin{column}{0.75\tttwi}
\begin{block}{UMAP Algorithm}
\benum
\item[]{\bf Input} $k$ number nearest neighbors, $d$, 
\item Find $k$-nearest neighbors
\item Construct (asymmetric) similarities $w_{ij}$, so that $\sum_jw_{ij}=\log_2 k$. $W=[w_{ij}]$.
\item Similarity matrix $S=W+W^T-W.*W^T$  
\item \myemph{Initialize embedding $\phi$ by {\sc LaplacianEigenmaps}}.
\item Optimize embedding.
 \item[] Iteratively for $n_{iter}$ steps
 \benum
 \item Sample an edge $ij$ with probability $\propto \exp{-d_{ij}}$
 \item Move $\phi_i$ towards $\phi_j$
 \item Sample a random $j'$ uniformly
 \item Move $\phi_i$ away from $\phi_{j'}$
% \item[]\myemph{\small Stochastic approximate \mydef{logistic regression} of $||\phi_i-\phi_j||$ on $d_{ij}$.}
\eenum
\item[]{\bf Output} $\phi$
  \eenum
  \end{block}
\end{column}
\begin{column}{0.2\tttwi}
\hfill\includegraphics[width=\picwi]{Figures-aspirin/ethanol_umap3D.png}\end{column}
\end{columns}

\vfill
\pause
\bit
\item t-SNE with appropriate choice of parameters can emulate UMAP \likecite{B\"ohm et al., 2022}
\eit

\end{frame}
%------------------------------------------------------------------------
\begin{frame}\frametitle{Embedding algorithms summary}
\begin{columns}
\begin{column}{0.5\tttwi}
  \myred{E-vector based} % minimize a functional
  \bit
\item \mydef{DiffusionMaps}
\item Isomap
\item LTSA \likecite{Zhang, Zha, 2004}
\item \ldots
\item[$+$] well studied, params better understood
\item[$-$] collapsed embeddings, \myemph{``horseshoes''} possible
%%% *** *** \item require embedding dimension $s\geq d$
  \eit
\end{column}
\pause
\begin{column}{0.5\tttwi}
  \myred{Repulsion based}
  \bit
\item t-SNE
\item UMAP
\item \ldots
\item[]
\item[$-$] heuristic, more parameters
\item[$+$] no collapsed embeddings
\item[]
%%% *** *** \item  $s=d$ intrinsic dimension
  \eit
\end{column}
\end{columns}

\pause
\vspace{2em}
  \bit
%\item All algorithms start from neighborhood graph + distance matrix $A$ (or similarity $S$)
%\item E-vector or repulsion based
\item  Embeddings are sensitive to
  \bit
\item neighborhoods scale \myblue{$\epsilon$} and type  \myblue{K-nn vs. spherical}
\item data non-uniformity 
\item aspect ratio (E-vector based)
  \eit
  \pause
\item Almost always embedding algorithms \myemph{distort} shape of data
  \eit
  
\end{frame}
%------------------------------------------------------------------------




% ----------------------------------------------------------------------
\subsection{Overview of {\tt SuperMan}}
% ----------------------------------------------------------------------
\begin{frame}{Implementation: SuperMan vs.\ scikit-learn}
	\begin{itemize}
		\item Performance:
		\begin{itemize}
			\item SuperMan: fast, real-time or semi-real-time
			\item scikit-learn: slower; may not complete on large data sets
		\end{itemize}
		\item Statistical and geometric features in SuperMan:
		\begin{itemize}
			\item neighbor selection ($k$-NN, radius $\varepsilon$)
			\item Riemannian metric estimation (RMetric)
			\item intrinsic dimension estimation (IES)
			\item other diagnostics and elementary statistics
		\end{itemize}
	\end{itemize}
\end{frame}

% ======================================================================
\section{SBI Validation Pipeline}
% ======================================================================
% ----------------------------------------------------------------------
\begin{frame}{SBI Validation Pipeline}
	\begin{itemize}
		\item Construct distance matrix and local statistics.
		\item Preprocessing I: Outlier removal.
		\item Assess coverage and fidelity.
		\item Preprocessing II: Uniform resampling.
		\item Construct the Laplacian and recompute local statistics.
		\item Estimate intrinsic dimension.
		\item Manifold Learning: \dm.
		\item Manifold Interpretation \& Visualization I: \ies and plotting.
		\item Manifold Interpretation \& Visualization II: \tslasso.
		\item Manifold Interpretation \& Visualization III: \rrelex.
	\end{itemize}
\end{frame}

% ----------------------------------------------------------------------
% 1. Construct distance matrix and local statistics
% ----------------------------------------------------------------------
\begin{frame}{Construct Distance Matrix and Local Statistics}
	
	\textbf{Optional Step: Sub-sampling for tractability}
	\begin{itemize}
		\item Many downstream operations scale superlinearly in $N$:
		\begin{itemize}
			\item pairwise distances / neighbor graphs,
			\item spectral decompositions of graph Laplacians,
			\item local statistics (kNN, radius counts, etc.),
			\item local covariance matrices and their spectral decompositions. 
		\end{itemize}
		\item If $N$ is very large:
		\begin{itemize}
			\item Randomly sub-sample to a size that:
			\begin{itemize}
				\item still captures the geometry of the data,
				\item fits within the available computational budget.
			\end{itemize}
			\item In general, one should use as many points as resources allow.
		\end{itemize}
	\end{itemize}
	
\end{frame}

\begin{frame}{Construct Distance Matrix and Local Statistics}
	
	\textbf{Main Step: Distance matrix and local statistics}
	\begin{itemize}
		\item Compute distances between latent points in $\X$.
		\item For each point $x \in \X$, compute local statistics:
		\begin{itemize}
			\item $\dk[x]$: distance to the $k$-th nearest neighbor for a range $k \in \{k_{\min}, \dots, k_{\max}\}$ or,
			\item $\nr[x]$: the number of neighbors within radius $r$ for a range $r \in \{r_{\min}, \dots, r_{\max}\}$.
		\end{itemize}
	\end{itemize}
	
	\textbf{Practical notes:}
	\begin{itemize}
		\item Store distances in sparse form.
		\item As we subsample the data, slice the already computed distance matrix.
		\item Reuse the same local statistics across multiple steps in the pipeline.
	\end{itemize}
\end{frame}


% ----------------------------------------------------------------------
% 2. Preprocessing I: Outlier removal
% ----------------------------------------------------------------------

\begin{frame}{Preprocessing I: Outlier Removal}
	\textbf{Motivation}
	\begin{itemize}
		\item Highly noisy or pathological points:
		\begin{itemize}
			\item are not representative of the underlying geometry or data distribution
			\item distort local geometry,
			\item destabilize spectral embeddings,
			\item bias intrinsic dimension estimates.
		\end{itemize}
		\item For this purpose, we want a \textbf{clean}(i.e. outlier free) subset $\X^{clean} \subseteq \X$.
	\end{itemize}
	\vspace{0.7em}
	\textbf{Recommended approach: Minimum Volume Sets (MVS)}
\end{frame}

% ======================================================================
\section{Possible Venues}
% ======================================================================
% ----------------------------------------------------------------------
\begin{frame}{Possible Venues for the Tutorial}
	\begin{itemize}
		\item Flatiron Institute — early January
		\item eScience Institute — discuss with D.~Beck
		\item Instats
	\end{itemize}
\end{frame}

% ======================================================================
\section{Ancillary Notes}
% ======================================================================
% ----------------------------------------------------------------------
\begin{frame}{Ancillary Notes}
	\begin{itemize}
		\item \textbf{Friday, Nov.\ 15}: Initial deadline for low-level stabilized code  
		\\(internal note: this was the deadline for the first seminar presentation;
		please communicate if plans change)
		\item \textbf{Monday, Dec.\ 15}: Initial deadline for tutorial materials
	\end{itemize}
\end{frame}

% ======================================================================
\bibliographystyle{ieeetr}
\bibliography{references.bib}
% ======================================================================
\end{document}
